\documentclass[a4paper,12pt]{extarticle}
\usepackage{geometry}
\usepackage[T1]{fontenc}
\usepackage[utf8]{inputenc}
\usepackage[english,russian]{babel}
\usepackage{amsmath}
\usepackage{amsthm}
\usepackage{amssymb}
\usepackage{fancyhdr}
\usepackage{setspace}
\usepackage{graphicx}
\usepackage{colortbl}
\usepackage{tikz}
\usepackage{pgf}
\usepackage{subcaption}
\usepackage{listings}
\usepackage{indentfirst}
\usepackage[
backend=biber,
style=numeric,
maxbibnames=99
]{biblatex}
\addbibresource{refs.bib}
\usepackage[colorlinks,citecolor=blue,linkcolor=blue,bookmarks=false,hypertexnames=true, urlcolor=blue]{hyperref} 
\usepackage{indentfirst}
\usepackage{mathtools}
\usepackage{booktabs}
\usepackage[flushleft]{threeparttable}
\usepackage{tablefootnote}

\usepackage{chngcntr} % нумерация графиков и таблиц по секциям
\counterwithin{table}{section}
\counterwithin{figure}{section}

\graphicspath{{graphics/}}%путь к рисункам

\makeatletter
% \renewcommand{\@biblabel}[1]{#1.} % Заменяем библиографию с квадратных скобок на точку:
\makeatother

\geometry{left=2.5cm}% левое поле
\geometry{right=1.0cm}% правое поле
\geometry{top=2.0cm}% верхнее поле
\geometry{bottom=2.0cm}% нижнее поле
\setlength{\parindent}{1.25cm}
\renewcommand{\baselinestretch}{1.5} % междустрочный интервал


\newcommand{\bibref}[3]{\hyperlink{#1}{#2 (#3)}} % biblabel, authors, year
\addto\captionsrussian{\def\refname{Список литературы (или источников)}} 

\renewcommand{\theenumi}{\arabic{enumi}}% Меняем везде перечисления на цифра.цифра
\renewcommand{\labelenumi}{\arabic{enumi}}% Меняем везде перечисления на цифра.цифра
\renewcommand{\theenumii}{.\arabic{enumii}}% Меняем везде перечисления на цифра.цифра
\renewcommand{\labelenumii}{\arabic{enumi}.\arabic{enumii}.}% Меняем везде перечисления на цифра.цифра
\renewcommand{\theenumiii}{.\arabic{enumiii}}% Меняем везде перечисления на цифра.цифра
\renewcommand{\labelenumiii}{\arabic{enumi}.\arabic{enumii}.\arabic{enumiii}.}% Меняем везде перечисления на цифра.цифра

\begin{document}
\begin{titlepage}
\newpage

\begin{center}
FEDERAL STATE AUTONOMOUS EDUCATIONAL INSTITUTION FOR\\
HIGHER PROFESSIONAL EDUCATION NATIONAL RESEARCH\\
UNIVERSITY\\
«HIGHER SCHOOL OF ECONOMICS»\\
\bigskip
Faculty of Computer Science
\end{center}

\vspace{2em}

\begin{center}
\underline{Marozau Leu}
\end{center}

\vspace{2em}

\begin{center}
Обнаружение эмоций на основе текста
\end{center}

\vspace{2em}

\begin{center}
Text-Based Emotion Detection
\end{center}

\vspace{4em}

\begin{center}
Qualification paper — Master of Science Dissertation\\
Field of study 01.04.02 «Applied Mathematics and Informatics»\\
Program: 
\end{center}

\vspace{6em}

\begin{flushleft}
Student\\
Marozau Leu
\end{flushleft}

\vspace{4em}

\begin{flushright}
Supervisor\\
Shirnin Alexander Andreevich
\end{flushright}

\vspace{\fill}

\begin{center}
Moscow, 2025
\end{center}

\end{titlepage}
% это титульный лист - выберите подходящий вам из имеющихся в проекте вариантов (kr - курсовая работа у 3 курса, vkr - выпускная квалификационная работа у 4 курса)
\newpage
\setcounter{page}{2}

{
	\hypersetup{linkcolor=black}
	\tableofcontents
}

\newpage

\section*{Abstract}

This thesis explores various approaches for multi-label emotion detection in text across multiple languages. We investigate traditional supervised approaches including BERT, SetFit, and Seq2Seq models, as well as a novel retrieval-augmented generation system called EmoRAG. Using the BRIGHTER dataset, which covers 28 languages including many low-resource ones, we demonstrate that our EmoRAG system achieves state-of-the-art performance without requiring extensive model training. This work contributes to the field of multilingual emotion recognition by providing a comparative analysis of different approaches and introducing an efficient, scalable method for detecting emotions across diverse languages.

\addcontentsline{toc}{section}{Abstract}

\section*{Keywords}
Deep learning, emotion detection, multilingual NLP, retrieval-augmented generation, low-resource languages, multi-label classification

\section{Introduction}

Emotions are fundamental to human communication and experience, coloring our interactions, decisions, and perceptions. The ability to detect and understand emotions in text has become increasingly important in natural language processing (NLP), with applications spanning various domains including customer service, mental health monitoring, content recommendation, social media analysis, and educational technology.

Unlike traditional sentiment analysis, which typically focuses on determining whether a text is positive, negative, or neutral, emotion detection aims to identify specific emotional states such as joy, sadness, anger, fear, surprise, and disgust. This fine-grained understanding of affective content enables more nuanced and human-like interactions between computational systems and users.

The development of effective multi-lingual, multi-label emotion detection systems faces several interconnected challenges. Linguistic diversity presents a significant hurdle, as human languages vary dramatically in their lexical, syntactic, and semantic structures, including how emotions are expressed. Data scarcity is particularly problematic for low-resource languages, which lack substantial labeled datasets for emotion detection. The multi-label nature of emotions adds complexity, as texts frequently express multiple emotions simultaneously. Cross-cultural variations in how emotions are expressed and interpreted further complicate the development of universally applicable systems. Finally, computational efficiency concerns arise when considering the impracticality of training separate models for thousands of languages.

This thesis investigates multi-lingual, multi-label emotion detection, exploring approaches that can operate effectively with limited labeled data, adapt to linguistic and cultural differences, and handle the inherent complexity of multi-label emotion classification. We compare traditional supervised approaches with newer paradigms like retrieval-augmented generation, examining their effectiveness across languages with varying resource availability.

The remainder of this thesis is organized as follows:

\textbf{Chapter 2: Related Work} reviews previous research in emotion detection, multi-label classification, multi-lingual NLP models, retrieval-augmented generation, few-shot learning, and large language models for emotion detection.

\textbf{Chapter 3: Data} describes the BRIGHTER dataset, including its creation, annotation process, and composition.

\textbf{Chapter 4: Methodology} presents our four approaches: BERT-based fine-tuning, SetFit few-shot learning, Seq2Seq generative models, and our novel EmoRAG system.

\textbf{Chapter 5: Experiments} outlines our experimental setup, evaluation metrics, and hyperparameter tuning strategies.

\textbf{Chapter 6: Results} analyzes our experimental findings, including performance comparisons across approaches, languages, and emotion categories.

\textbf{Chapter 7: Discussion} interprets our results, comparing the strengths and weaknesses of each approach.

\textbf{Chapter 8: Conclusion and Future Work} summarizes key findings and suggests directions for future research.

\section{Related Work}

This chapter reviews key literature relevant to multi-lingual, multi-label emotion detection, focusing on four areas: traditional approaches to emotion detection, multi-label classification techniques, large language models, and retrieval-augmented generation systems.

\subsection{Emotion Detection in Text}

Emotion detection in text has evolved significantly over the past two decades. Early work by \cite{strapparava2007semeval} introduced one of the first datasets for emotion recognition in English text, defining the task as detecting Ekman's six basic emotions: joy, sadness, anger, fear, surprise, and disgust. This work established the foundation for most subsequent research in the field.

Classical approaches to emotion detection included lexicon-based methods and traditional machine learning techniques. \cite{mohammad2013crowdsourcing} developed the NRC Emotion Lexicon, mapping English words to emotions. These approaches provide strong baselines but struggle with contextual understanding and required extensive manual annotation for each language.

The advent of deep learning brought significant advances to emotion detection. \cite{felbo2017using} introduced DeepMoji, leveraging distant supervision from emojis to pre-train models for emotion recognition. More recently, transformer-based approaches have set new benchmarks. \cite{demszky2020goemotions} created GoEmotions, a large-scale dataset with 27 emotion categories, and demonstrated strong performance with BERT-based models.

\subsection{Multi-label Classification and Language Models}

Multi-label classification, where instances can belong to multiple categories simultaneously, is inherently relevant to emotion detection as texts often express multiple emotions at once. 

Classical approaches to multi-label classification include binary relevance, which trains independent classifiers for each label, and label powerset, which transforms the problem into a multi-class classification task with each unique label combination as a separate class. \cite{read2011classifier} introduced classifier chains, which model inter-label dependencies by building a sequence of binary classifiers, demonstrating significant improvements over binary relevance methods.

The evolution of language models has dramatically impacted text classification tasks, including emotion detection. BERT \cite{devlin2019bert} and its multilingual variant mBERT demonstrated strong performance across languages and tasks through contextual representations and pre-training on massive corpora. XLM-RoBERTa \cite{conneau2020unsupervised} further improved cross-lingual capabilities by training on 100 languages with 2.5 times more data than mBERT.

Large Language Models (LLMs) like GPT-3 \cite{brown2020language} and its successors have shown remarkable few-shot capabilities through in-context learning, where models make predictions based on examples provided in the prompt. This paradigm shift has significant implications for multi-lingual emotion detection, potentially reducing the need for extensive labeled data in each language.

For multi-label settings, deep learning approaches have shown promise. \cite{nam2014large} demonstrated that using cross-entropy loss with a sigmoid activation function per label outperforms ranking-based loss functions in neural networks. 

\subsection{Few-shot Learning in NLP}

Few-shot learning has become increasingly important for addressing data scarcity in NLP, particularly for low-resource languages. \cite{brown2020language} demonstrated the remarkable few-shot capabilities of large language models through in-context learning, where a model makes predictions based on a few examples provided in the prompt.

Building on this foundation, \cite{gao2021making} showed that carefully constructed prompts with strategically selected examples can dramatically improve few-shot performance on various NLP tasks, including classification. For more specialized applications, \cite{tunstall2022efficient} developed SetFit, a few-shot learning approach that combines contrastive learning with classification fine-tuning. By leveraging sentence transformers and efficient pair-wise training, SetFit achieves strong performance with as few as 8 examples per class.

The selection of examples for few-shot learning significantly impacts performance. \cite{liu2022few} showed that retrieving examples based on semantic similarity to the test instance outperforms random selection, especially for complex tasks. Cross-lingual few-shot learning presents additional challenges. 

Despite these advances, few-shot learning in multi-label settings remains challenging. \cite{hou2022few} addressed this gap with a specialized few-shot learning approach for multi-label classification that captures label correlations even with limited examples.

\subsection{Retrieval-Augmented Generation (RAG)}

Retrieval-Augmented Generation (RAG) has emerged as a powerful paradigm for enhancing language model capabilities with external knowledge. \cite{lewis2020retrieval} introduced the original RAG framework, combining dense retrieval with sequence-to-sequence models to generate outputs conditioned on relevant retrieved documents.

While RAG was initially developed for generation tasks, its application to classification has shown promising results. \cite{gao2024retrieval} adapted RAG for text classification, demonstrating that retrieved examples can serve as few-shot demonstrations to guide in-context learning. Their approach achieved competitive performance on various classification benchmarks without requiring task-specific fine-tuning.

For multi-lingual contexts, \cite{shi2023replug} developed Cross-Lingual RAG, which leverages a shared dense retrieval space across languages to retrieve relevant documents for low-resource languages. This approach is particularly valuable for emotion detection across diverse languages, as it enables knowledge transfer from high-resource to low-resource languages.

The selection of retrieved documents significantly impacts RAG performance. \cite{gao2023retrieval} introduced techniques for dynamic retrieval that adapt to the specific needs of each query, showing that adaptive retrieval strategies outperform static approaches. In classification contexts, \cite{singh2022flare} demonstrated that RAG can mitigate the challenge of domain shift by retrieving examples similar to the test instance, regardless of their source domain.

\subsection{Research Gaps and Opportunities}

Our review of the literature reveals several important gaps that this thesis aims to address:

\begin{enumerate}

    \item \textbf{Few-shot Learning for Low-resource Languages}: Most emotion detection systems require substantial labeled data, which is unavailable for many languages. Few studies have explored how large language models can effectively perform emotion detection with minimal examples across diverse languages.
    
    \item \textbf{Ensemble Methods for Multi-label Emotion Detection}: While ensemble approaches have shown promise in various NLP tasks, their application to multi-label emotion detection, particularly in multi-lingual settings, remains underexplored. There is limited research on how to effectively combine multiple LLM predictions to improve performance across emotions and languages.
    
    \item \textbf{Retrieval-Augmented Classification}: While RAG has shown promise for generation tasks, its application to multi-label classification, particularly emotion detection, remains underexplored.

    \item \textbf{Balancing Performance Across Languages}: Existing approaches often show significant performance disparities between high-resource and low-resource languages. Few studies have systematically addressed how to design systems that maintain consistent performance across linguistically diverse languages.
    
    \item \textbf{Low-Resource Adaptability}: Few studies have explicitly addressed how emotion detection approaches can be adapted for low-resource languages with minimal labeled data.

    \item \textbf{Capturing Emotion Co-occurrence Patterns}: The multi-label nature of emotion detection presents unique challenges, as emotions often co-occur in complex patterns. Limited research exists on how few-shot LLM approaches can effectively model these interdependencies without extensive training data.
    
\end{enumerate}

\section{Data}
\subsection{Overview}

\begin{figure}[h]
    \centering
    \includegraphics[width=0.5\textwidth]{brighter_languages.png}
    \caption{Languages in the BRIGHTER dataset}
    \label{fig:brighter_languages}
\end{figure}

\begin{figure}[h]
    \centering
    \includegraphics[width=0.8\textwidth]{brighter_examples.png}
    \caption{Examples from the BRIGHTER dataset}
    \label{fig:brighter_examples}
\end{figure}

The BRIGHTER dataset \cite{muhammad2025brighterbridginggaphumanannotated} is a comprehensive, multi-labeled, multilingual resource for textual emotion recognition, consisting of over 100,000 annotated examples across 28 languages and 7 language families. Designed to address the stark resource imbalance in emotion research, BRIGHTER prioritizes low-resource languages from Africa, Asia, Eastern Europe, and Latin America, while also including mid- and high-resource languages such as English, German, and Portuguese.

Each instance in BRIGHTER is manually annotated by fluent speakers, with annotations covering six core perceived emotions — \textit{joy}, \textit{sadness}, \textit{anger}, \textit{fear}, \textit{surprise}, and \textit{disgust} — and includes an additional \textit{neutral} label when no emotion is expressed. Annotations are multi-label and span four emotion intensity levels (0 to 3).

The dataset aggregates data from diverse sources, including social media platforms (e.g., Reddit, YouTube, Weibo), speeches, literature, news, and even machine-generated examples with human post-editing. A detailed breakdown of these sources, annotator counts, and splits per language is given in the original BRIGHTER paper, with representative examples shown in Figure~\ref{fig:brighter_examples} and the language family distribution illustrated in Figure~\ref{fig:brighter_languages}.

\subsection{Data Splits}
We use the official BRIGHTER dataset splits for all experiments:

\begin{itemize}
\item \textbf{Training Set}: Approximately 70\% of the data, used for supervised model training.
\item \textbf{Validation Set}: Approximately 15\% of the data, used for early stopping and hyperparameter tuning.
\item \textbf{Test Set}: Approximately 15\% of the data, reserved for final evaluation.
\end{itemize}

\subsection{Dataset Challenges}

\begin{figure}[h]
    \centering
    \includegraphics[width=1\textwidth]{brighter_label_distribution.png}
    \caption{Label distribution in the BRIGHTER dataset}
    \label{fig:brighter_label_distribution}
\end{figure}

While BRIGHTER is a rich and diverse resource, it also introduces several challenges that motivated our modeling choices:

Label Distribution: The dataset exhibits long-tailed distributions for both emotions and languages, as illustrated in Figure~\ref{fig:brighter_label_distribution}. Some languages lack examples for certain emotions (e.g., no \textit{disgust} in English, no \textit{surprise} in Afrikaans), and neutral examples vary greatly in frequency.

Linguistic Coverage: BRIGHTER spans typologically diverse languages with varying scripts (Latin, Arabic, Devanagari, Cyrillic), making it a strong benchmark for cross-lingual generalization.

Annotation Density: Most languages have between 3 and 10 annotators per example, but annotation intensity and agreement levels vary. Final labels were aggregated using a combination of per-emotion thresholding and average intensity, ensuring consistent quality across languages.

These considerations led us to explore both supervised and retrieval-augmented methods, with the aim of improving generalization under low-resource, high-diversity conditions.

\section{Methodology}

This chapter presents the various approaches we explored for multi-lingual, multi-label emotion detection using the BRIGHTER dataset. We investigate both traditional supervised learning approaches (BERT, SetFit, and Seq2Seq) and a novel retrieval-augmented generation (RAG) approach called EmoRAG. Each approach represents a different paradigm in machine learning for text classification:

\begin{enumerate}
\item \textbf{BERT-based Approach}: Fine-tune a pre-trained language model with a classification head for multi-label prediction
\item \textbf{SetFit Approach}: A few-shot learning technique that combines contrastive learning with standard classification techniques
\item \textbf{Seq2Seq Approach}: Framing emotion detection as a text generation task
\item \textbf{EmoRAG System}: A novel retrieval-augmented generation system that leverages few-shot learning without parameter updates
\end{enumerate}

The following sections describe each approach in detail, including model architecture, training procedures, and implementation specifics.

\subsection{BERT-based Approach}

For prediction, we apply a sigmoid function to the model outputs and use a threshold of 0.5 to convert probabilities to binary predictions:

The BERT-based approach provides a strong baseline for emotion detection, leveraging the powerful contextual representations of transformer models. However, it may struggle with low-resource languages where pre-training data is limited.

\subsection{Seq2Seq Approach}

The Seq2Seq approach reframes multi-label emotion detection as a text generation task, where the model generates a comma-separated list of emotions present in the input text. This approach leverages the generative capabilities of encoder-decoder transformer models to directly produce structured outputs rather than independent binary classifications.

This approach is particularly interesting because of the large number of available pre-trained encoder-decoder models.

Some examples of pre-trained encoder-decoder models are:

\textbf{mT5} \cite{xue2021mt5massivelymultilingualpretrained}: A multilingual variant of T5 (Text-to-Text Transfer Transformer) pre-trained on 101 languages. 

\textbf{BART} \cite{lewis2019bartdenoisingsequencetosequencepretraining}: A denoising autoencoder for pretraining sequence-to-sequence models. 

\textbf{mBART} \cite{liu2020multilingualdenoisingpretrainingneural}: A multilingual sequence-to-sequence denoising auto-encoder pre-trained on 25 languages.

The Seq2Seq approach transforms the multi-label classification problem into a text generation task:

\textbf{Input}: The original text to be classified. 

\textbf{Output}: A comma-separated list of emotion labels present in the text (e.g., "joy,surprise,fear").

This format allows the model to learn the relationships between emotions naturally through the generation process, potentially capturing label co-occurrences and dependencies more effectively than independent binary classifiers.

\subsection{SetFit Approach}

SetFit (Sentence Transformer Fine-tuning) is a novel few-shot learning method introduced by \cite{tunstall2022efficient} that combines contrastive learning with standard classification techniques. It was designed to achieve strong performance with limited labeled data, making it particularly suitable for multi-lingual emotion detection where labeled examples may be scarce for low-resource languages.

The SetFit approach involves two stages. 

The first stage is the \textbf{Contrastive Learning Stage}, where a sentence transformer model is fine-tuned using contrastive learning on sentence pairs derived from labeled examples. 

The second stage is the \textbf{Classification Stage}, where a classifier, typically a linear model, is trained on the embeddings produced by the fine-tuned sentence transformer.

Figure~\ref{fig:setfit_training} illustrates the SetFit training process.

\begin{figure}[h]
    \centering
    \includegraphics[width=0.8\textwidth]{setfit.png}
    \caption{SetFit training process}
    \label{fig:setfit_training}
\end{figure}

The key innovation of SetFit is its ability to leverage the power of sentence transformers for few-shot learning without requiring extensive labeled data or computationally expensive prompt-based approaches. By fine-tuning the sentence embeddings directly on the task-specific data, SetFit can adapt pre-trained embeddings to better represent the nuances of emotion detection across languages.

\subsection{EmoRAG System}

EmoRAG (Emotion Retrieval-Augmented Generation) is a novel system developed for multi-label emotion detection that reframes the problem through the lens of retrieval-augmented generation. Rather than relying on model fine-tuning, EmoRAG leverages the knowledge captured in labeled examples retrieved at inference time, enabling multilingual, few-shot emotion classification without updating model parameters.

\subsubsection{Pipeline Overview}

The EmoRAG architecture follows a four-stage pipeline, illustrated in Figure~\ref{fig:emorag_pipeline}:

\begin{enumerate}
\item \textbf{Database Construction}: The system first indexes the labeled examples from the BRIGHTER training dataset as a retrieval corpus.
\item \textbf{Retrieval}: An n-gram or embedding-based retriever fetches the top-K most similar examples to a given input.
\item \textbf{Generation}: The retrieved examples are used as few-shot prompts for a set of pre-trained large language models (LLMs) to produce emotion predictions.
\item \textbf{Aggregation}: The individual model predictions are combined via a learned or heuristic aggregation strategy (e.g., majority vote, label-weighted averaging) to produce the final multi-label output.
\end{enumerate}


\begin{figure}[h]
    \centering
    \includegraphics[width=0.8\textwidth]{emorag.png}
    \caption{The EmoRAG pipeline includes a retriever, a set of LLMs, and an aggregation module. Retrieved labeled examples are used as prompts to generate multi-label predictions.}
    \label{fig:emorag_pipeline}
\end{figure}

\subsubsection{Retriever Component}

We experimented with two retrieval strategies:
\begin{itemize}
\item \textbf{N-gram Overlap}: Particularly effective for low-resource languages, this method retrieves examples based on surface lexical similarity.
\item \textbf{Embedding-based Retrieval}: A dense vector retriever using sentence-level embeddings that supports multilingual representations. For our experiments, we used the BGE-M3 embedding model \cite{chen2024bgem3embeddingmultilingualmultifunctionality}, which achieved state-of-the-art performance on the multilingual subset of the MTEB benchmark \cite{muennighoff2023mtebmassivetextembedding}.
\end{itemize}

The number of retrieved samples K is fixed per language type: 30 for low-resource languages and 100 for high-resource ones.

\subsubsection{Generator Models}

For generation, EmoRAG uses a diverse ensemble of LLMs including:
\texttt{Llama-3.1-70B \cite{grattafiori2024llama3herdmodels}}, \texttt{Qwen2.5-72B-Instruct \cite{yang2024qwen2technicalreport}}, \texttt{Gemma-2-27B-it \cite{gemmateam2024gemma2improvingopen}}, \texttt{GPT-4o-mini \cite{openai2024gpt4omini}}
Each LLM receives the same prompt structure, written in English and formatted to generate emotion predictions in JSON format:
\begin{verbatim}
{
"anger": bool,
"fear": bool,
"joy": bool,
"sadness": bool,
"surprise": bool,
"disgust": bool
}
\end{verbatim}

\subsubsection{Aggregation Strategies}

To aggregate predictions from multiple LLMs, we evaluate five strategies. 

The \textbf{Single model} approach simply uses predictions from one LLM (e.g., GPT-4o-mini) without any aggregation. 

\textbf{Majority Vote} gives each model an equal vote for each emotion label, with the final prediction determined by the majority decision (threshold > 0.5). 

In \textbf{Macro/Micro-F1 Weighted Voting}, models' votes are weighted by their macro or micro F1 scores on the validation set for the specific language, giving more influence to better-performing models. 

\textbf{Label-specific F1 Voting} weights each model's vote by its F1 score for the specific emotion label and language, allowing models with higher performance on particular emotions to have more influence on those predictions. 

Finally, \textbf{LLM-based Aggregation} uses one LLM to analyze and aggregate the outputs from other models, potentially capturing more nuanced patterns in the predictions.

\section{Experiments}

\subsection{Experimental Setup}

Our experiments were conducted in two phases: first, a comprehensive evaluation of all approaches on English data to establish baseline performance, followed by an extensive multilingual evaluation focusing on our EmoRAG system across all 28 languages in the BRIGHTER dataset.

\subsubsection{Computing Infrastructure}

All experiments were conducted on the HSE University High-Performance Computing (HPC) Cluster "cHARISMa". This infrastructure provided essential computational resources for our large-scale experiments, particularly for training and evaluating transformer-based models. The cluster includes specialized nodes with NVIDIA Tesla V100 32GB GPUs and newer nodes with NVIDIA A100 80GB GPUs, which were crucial for training our larger models. For the most computationally intensive experiments, we utilized Type C computing nodes equipped with 4 NVIDIA Tesla V100 32GB GPUs with NVLink and 768GB RAM.

\subsubsection{Model Configurations}

For each approach, we employed the following configurations:

\begin{itemize}
\item \textbf{BERT-based Approach}: We fine-tuned XLM-RoBERTa-large (550M parameters) with a classification head for multi-label prediction. The model was initialized with pre-trained weights and fine-tuned on our emotion detection task with a learning rate of 2e-5 and batch size of 16.

\item \textbf{SetFit Approach}: We used the multilingual MPNet base model as our sentence transformer backbone, with a multi-output classification strategy. For few-shot learning, we sampled 8 examples per emotion category, balanced across languages when possible.

\item \textbf{Seq2Seq Approach}: We employed mT5-large (1.2B parameters) as our primary sequence-to-sequence model, fine-tuned with a learning rate of 5e-5 and a batch size of 8. We used a maximum sequence length of 512 tokens for both input and output.

\item \textbf{EmoRAG System}: Our retrieval component used the BGE-M3 embedding model for dense retrieval and n-gram retrieval, with K=30 for low-resource languages and K=100 for high-resource languages. The generator ensemble included Llama-3.1-70B, Qwen2.5-72B-Instruct, Gemma-2-27B-it, and GPT-4o-mini.
\end{itemize}

\subsubsection{Training Process}

For the supervised approaches (BERT, SetFit, and Seq2Seq), we trained models for 10 epochs with early stopping based on validation loss. The BERT and Seq2Seq models were trained using AdamW optimizer with a linear learning rate scheduler and 10\% warmup steps. 
For SetFit, we used a two-stage training process: contrastive learning followed by classifier training.

The EmoRAG system required no training in the traditional sense, as it leverages pre-trained models and retrieval mechanisms. 

All training runs were managed using the Slurm workload manager on the HPC cluster, with typical training times ranging from 2 hours for SetFit to 24 hours for the Seq2Seq approach on the full dataset.

For the English-only experiments, we used validation set for evaluation of different approaches.

Then, for EmoRAG system we provide evaluation both on the validation and test sets.

\subsection{Evaluation Metrics}

Our primary evaluation metrics were F1-micro and F1-macro scores, which are particularly suitable for multi-label classification tasks:

\textbf{F1-micro} calculates metrics globally by counting the total true positives, false negatives, and false positives across all instances. This metric gives equal weight to each sample and is therefore influenced more by common labels and high-resource languages.

\textbf{F1-macro} calculates metrics for each label separately and then takes the unweighted mean. This gives equal importance to each label regardless of its frequency, making it more suitable for evaluating performance on imbalanced datasets.

\section{Results}

\subsection{Overall Performance}

Our experiments revealed significant performance differences across the various approaches to emotion detection. Table \ref{tab:english_comparison} presents a comprehensive comparison of all methods on the English subset of the BRIGHTER dataset.

\begin{table}[h]
\centering
\begin{tabular}{lcc}
\toprule
\textbf{Method} & \textbf{F1-micro} & \textbf{F1-macro} \\
\midrule
BERT (bert-large-cased) & 0.7145 & 0.6468 \\
SetFit (bge-m3) & 0.6943 & 0.5591 \\
Seq2Seq (bart-large-cnn) & 0.5126 & 0.4566 \\
GPT-4 (zero-shot) & 0.5959 & 0.6164 \\
\midrule
\multicolumn{3}{c}{\textbf{EmoRAG Variants}} \\
\midrule
Llama-3.1-70B (few-shot, bge-m3) & 0.7499 & 0.7239 \\
Qwen2.5 (few-shot, bge-m3) & 0.7790 & 0.7749 \\
GPT-4o-mini (few-shot, bge-m3) & 0.8071 & 0.8026 \\
GPT-4o-mini (few-shot, ngram) & 0.7810 & 0.7700 \\
EmoRAG (majority vote) & 0.8080 & 0.8010 \\
EmoRAG (majority vote macro) & 0.8230 & 0.8200 \\
EmoRAG (label-specific F1) & \textbf{0.8210} & \textbf{0.8180} \\
\bottomrule
\end{tabular}
\caption{Performance comparison of different approaches on English language subset of the validation set}
\label{tab:english_comparison}
\end{table}

The EmoRAG system with label-specific F1 weighting achieved the highest performance, with an F1-micro score of 0.821 and F1-macro score of 0.818, significantly outperforming all other approaches. Among the traditional supervised methods, BERT performed best with F1-micro of 0.7145, followed by SetFit at 0.6943. The Seq2Seq approach showed the weakest performance among supervised methods, with an F1-micro of only 0.5126.

Notably, all EmoRAG variants outperformed the supervised approaches, with even the single-model GPT-4o-mini achieving an F1-micro of 0.8071, demonstrating the effectiveness of retrieval-augmented generation for this task. The ensemble-based aggregation strategies further improved performance, with label-specific F1 weighting providing the best results.

In terms of efficiency, the supervised approaches required substantial training time (24 hours for Seq2Seq, 8 hours for BERT, and 2 hours for SetFit on our hardware), while EmoRAG required no training but had higher inference costs due to the multiple LLM calls. The average inference time per sample was 0.05 seconds for BERT, 0.12 seconds for SetFit, 0.3 seconds for Seq2Seq, and 2.5 seconds for EmoRAG with four models.

\subsection{Performance by Language}

\begin{figure}[h]
    \centering
    \includegraphics[width=1\textwidth]{emorag_by_language.png}
    \caption{EmoRAG performance across all 28 languages in the BRIGHTER dataset}
    \label{fig:emorag_by_language}
\end{figure}


Table \ref{tab:multilingual_performance} presents the performance of our best EmoRAG system across all 28 languages in the BRIGHTER dataset on the validation set, showing F1-micro/F1-macro scores for each language and aggregation method.

Table \ref{tab:test_metrics} presents the performance of our best EmoRAG system across all 28 languages in the BRIGHTER dataset on the test set, showing F1-micro/F1-macro scores for each language and aggregation method.

Figure \ref{fig:emorag_by_language} shows the performance of our best EmoRAG system across all 28 languages in the BRIGHTER dataset on the validation set, showing F1-micro/F1-macro scores for the best performing model for each language.

\begin{table*}[!th]
\begin{center}
    
    \footnotesize
    \resizebox{\textwidth}{!}{
    \begin{tabular}{@{}lccccccccc@{}}
    \toprule
    \textbf{Language} & \textbf{llama-3.1-70b} & \textbf{qwen2.5-70b} & \textbf{gpt-4o-mini} & \textbf{gpt-4o-mini-ngram} & \textbf{gemma29b} & \textbf{gemma29b\_ngram} & \textbf{majority\_vote} & \textbf{majority\_vote\_macro} & \textbf{majority\_vote\_by\_label\_f1} \\ \midrule
    amh & 0.534/0.448 & - & \textbf{0.637/0.503} & 0.633/0.493 & 0.609/0.488 & 0.582/0.474 & 0.659/0.535 & 0.659/0.535 & 0.655/0.539 \\
    arq & 0.584/0.575 & 0.623/0.597 & 0.613/0.596 & \textbf{0.663/0.655} & 0.614/0.589 & 0.578/0.531 & 0.645/0.615 & 0.653/0.665 & 0.687/0.677 \\
    ary & 0.542/0.490 & 0.540/0.485 & 0.552/0.499 & \textbf{0.576/0.512} & 0.575/0.521 & 0.584/0.484 & 0.607/0.526 & 0.526/0.599 & 0.616/0.540 \\
    afr & 0.560/0.444 & \textbf{0.629/0.527} & 0.662/0.567 & 0.646/0.572 & 0.584/0.481 & 0.484/0.398 & 0.601/0.494 & 0.546/0.646 & 0.662/0.557 \\
    chn & 0.676/0.603 & 0.589/0.570 & 0.698/0.579 & \textbf{0.748/0.604} & 0.693/0.572 & 0.709/0.543 & 0.749/0.642 & 0.652/0.757 & 0.759/0.659 \\
    deu & 0.745/0.588 & 0.521/0.499 & \textbf{0.745/0.694} & 0.738/0.662 & 0.632/0.559 & 0.659/0.593 & 0.738/0.659 & 0.672/0.741 & 0.752/0.695 \\
    eng & 0.735/0.726 & 0.779/0.775 & 0.807/0.803 & 0.770/0.781 & 0.769/0.759 & 0.720/0.723 & 0.801/0.808 & 0.823/0.820 & \textbf{0.821/0.818} \\
    esp & 0.751/0.744 & 0.788/0.778 & 0.793/0.785 & 0.799/0.793 & 0.778/0.772 & 0.782/0.778 & 0.786/0.778 & 0.807/0.812 & \textbf{0.813/0.809} \\
    hau & 0.610/0.602 & 0.607/0.598 & 0.669/0.662 & \textbf{0.696/0.687} & 0.682/0.676 & 0.698/0.689 & 0.735/0.728 & 0.734/0.738 & 0.735/0.731 \\
    hin & 0.780/0.791 & 0.707/0.728 & 0.805/0.803 & 0.811/0.812 & 0.796/0.799 & 0.798/0.806 & 0.838/0.842 & 0.833/0.830 & \textbf{0.842/0.849} \\
    ibo & 0.531/0.486 & 0.502/0.452 & 0.572/0.514 & 0.564/0.499 & 0.574/0.508 & 0.574/0.520 & 0.609/0.532 & 0.534/0.608 & \textbf{0.614/0.550} \\
    kin & 0.443/0.385 & 0.443/0.382 & 0.555/0.491 & \textbf{0.576/0.489} & 0.477/0.404 & 0.514/0.466 & 0.589/0.515 & 0.501/0.570 & 0.575/0.512 \\
    mar & 0.874/0.883 & 0.904/0.908 & 0.937/0.939 & 0.937/0.939 & 0.883/0.883 & 0.897/0.900 & 0.942/0.946 & 0.935/0.931 & \textbf{0.943/0.947} \\
    orm & 0.467/0.369 & 0.521/0.415 & 0.552/0.455 & \textbf{0.607/0.501} & 0.519/0.404 & 0.488/0.362 & 0.585/0.446 & 0.493/0.608 & 0.608/0.488 \\
    pcm & 0.532/0.508 & 0.573/0.535 & 0.599/0.542 & \textbf{0.628/0.573} & 0.608/0.572 & 0.585/0.548 & 0.621/0.574 & 0.590/0.633 & 0.638/0.591 \\
    ptbr & 0.686/0.547 & 0.662/0.569 & 0.731/0.633 & 0.707/0.603 & 0.726/0.617 & 0.710/0.525 & 0.766/0.626 & 0.658/0.760 & \textbf{0.766/0.645} \\
    ptmz & 0.454/0.456 & 0.539/0.532 & 0.515/0.484 & 0.478/0.443 & 0.521/0.486 & 0.494/0.445 & 0.565/0.558 & 0.543/0.552 & \textbf{0.565/0.558} \\
    ron & 0.758/0.749 & 0.745/0.726 & 0.756/0.741 & 0.778/0.763 & 0.745/0.719 & 0.754/0.724 & 0.773/0.751 & 0.771/0.790 & \textbf{0.794/0.774} \\
    rus & 0.835/0.836 & 0.861/0.857 & 0.839/0.833 & 0.812/0.806 & 0.841/0.834 & 0.824/0.817 & 0.879/0.877 & 0.881/0.883 & \textbf{0.880/0.880} \\
    som & 0.361/0.296 & 0.379/0.338 & 0.518/0.469 & 0.528/0.491 & 0.426/0.381 & 0.428/0.382 & 0.494/0.420 & 0.464/0.514 & \textbf{0.519/0.477} \\
    sun & 0.674/0.496 & 0.707/0.491 & 0.734/0.596 & 0.757/0.612 & 0.708/0.532 & 0.733/0.565 & 0.754/0.537 & 0.564/0.750 & \textbf{0.757/0.614} \\
    swa & 0.357/0.329 & 0.376/0.345 & 0.391/0.366 & 0.416/0.401 & 0.401/0.366 & 0.407/0.372 & 0.435/0.396 & 0.401/0.435 & \textbf{0.440/0.409} \\
    swe & 0.684/0.475 & 0.680/0.502 & 0.709/0.528 & 0.708/0.529 & 0.699/0.518 & 0.671/0.501 & 0.734/0.555 & 0.547/0.727 & \textbf{0.736/0.582} \\
    tat & 0.652/0.611 & 0.663/0.631 & 0.712/0.671 & 0.702/0.660 & 0.669/0.634 & 0.637/0.592 & 0.727/0.673 & 0.688/0.732 & \textbf{0.749/0.710} \\
    tir & - & - & 0.377/0.321 & 0.384/0.319 & - & - & 0.322/0.263 & 0.321/0.377 & \textbf{0.397/0.342} \\
    ukr & 0.521/0.512 & 0.601/0.579 & 0.581/0.567 & 0.550/0.537 & 0.587/0.553 & 0.535/0.469 & 0.622/0.611 & 0.621/0.625 & \textbf{0.634/0.621} \\
    vmw & 0.158/0.145 & 0.261/0.184 & 0.300/0.211 & 0.226/0.158 & 0.246/0.206 & 0.186/0.159 & 0.190/0.140 & 0.180/0.230 & \textbf{0.257/0.205} \\
    yor & 0.354/0.255 & 0.415/0.300 & 0.474/0.374 & 0.506/0.420 & 0.436/0.317 & 0.472/0.347 & 0.564/0.443 & 0.423/0.532 & \textbf{0.564/0.443} \\ \midrule
    \textbf{Average} & \textbf{0.563/0.515} & \textbf{0.590/0.556} & \textbf{0.631/0.590} & \textbf{0.641/0.601} & \textbf{0.617/0.576} & \textbf{0.607/0.566} & \textbf{0.661/0.617} & \textbf{0.646/0.634} & \textbf{0.678/0.634} \\ \bottomrule
    \end{tabular}
    }
\end{center}
\caption{Development set F1-micro/F1-macro scores for each language and model. The best model for each language is highlighted in bold.}
\label{tab:multilingual_performance}
\end{table*}

\begin{table*}[th!]
\centering
\footnotesize

\resizebox{.99\textwidth}{!}{

\begin{tabular}{@{}lcccccc@{}}
\toprule
\textbf{Language} & \textbf{Language Code} & \textbf{Best Model} & \textbf{Dev F1 Micro} & \textbf{Dev F1 Macro} & \textbf{Test F1 Micro} & \textbf{Test F1 Macro} \\ \midrule
Afrikaans & afr & majority\_vote\_by\_label\_f1 & 0.662 & 0.557 & 0.7153 & 0.667 \\
Amharic & amh & gpt-4o-mini & 0.637 & 0.503 & 0.6613 & 0.5578 \\
German & deu & gpt-4o-mini & 0.745 & 0.694 & 0.2694 & 0.2156 \\
English & eng & majority\_vote\_by\_label\_f1 & 0.821 & 0.818 & 0.8066 & 0.7885 \\
Spanish & esp & majority\_vote\_by\_label\_f1 & 0.813 & 0.809 & 0.8204 & 0.8174 \\
Hindi & hin & majority\_vote\_by\_label\_f1 & 0.842 & 0.849 & 0.8658 & 0.8661 \\
Marathi & mar & majority\_vote\_by\_label\_f1 & 0.943 & 0.947 & 0.8559 & 0.864 \\
Oromo & orm & gpt-4o-mini-ngram & 0.607 & 0.501 & 0.6023 & 0.4903 \\
Portuguese (Brazil) & ptbr & majority\_vote\_by\_label\_f1 & 0.766 & 0.645 & 0.4809 & 0.372 \\
Russian & rus & majority\_vote\_by\_label\_f1 & 0.880 & 0.880 & 0.8829 & 0.8794 \\
Somali & som & majority\_vote\_by\_label\_f1 & 0.519 & 0.477 & 0.5422 & 0.5082 \\
Sundanese & sun & gpt-4o-mini-ngram & 0.757 & 0.612 & 0.7256 & 0.5294 \\
Tatar & tat & majority\_vote\_by\_label\_f1 & 0.749 & 0.710 & 0.7884 & 0.7763 \\
Tigrinya & tir & majority\_vote\_by\_label\_f1 & 0.397 & 0.342 & 0.2597 & 0.2044 \\
Arabic (Algerian) & arq & majority\_vote\_by\_label\_f1 & 0.687 & 0.677 & 0.5464 & 0.5203 \\
Arabic (Moroccan) & ary & gpt-4o-mini-ngram & 0.576 & 0.512 & 0.4089 & 0.3701 \\
Chinese (Mandarin) & chn & gpt-4o-mini-ngram & 0.748 & 0.604 & 0.7416 & 0.6252 \\
Hausa & hau & majority\_vote\_by\_label\_f1 & 0.735 & 0.731 & 0.7039 & 0.6954 \\
Kinyarwanda & kin & gpt-4o-mini-ngram & 0.576 & 0.489 & 0.6167 & 0.5627 \\
Nigerian Pidgin & pcm & majority\_vote\_by\_label\_f1 & 0.638 & 0.591 & 0.6416 & 0.5993 \\
Portuguese (Mozambique) & ptmz & majority\_vote\_by\_label\_f1 & 0.565 & 0.558 & 0.535 & 0.4927 \\
Swahili & swa & majority\_vote\_by\_label\_f1 & 0.440 & 0.409 & 0.43 & 0.3856 \\
Swedish & swe & majority\_vote\_by\_label\_f1 & 0.736 & 0.582 & 0.6353 & 0.4926 \\
Ukrainian & ukr & majority\_vote\_by\_label\_f1 & 0.634 & 0.621 & 0.638 & 0.6161 \\
Emakhuwa & vmw & gpt-4o-mini & 0.300 & 0.211 & 0.2556 & 0.2157 \\
Yoruba & yor & majority\_vote\_by\_label\_f1 & 0.564 & 0.443 & 0.5257 & 0.3818 \\
Igbo & ibo & majority\_vote\_by\_label\_f1 & 0.614 & 0.550 & 0.6125 & 0.5379 \\
Romanian & ron & majority\_vote\_by\_label\_f1 & 0.794 & 0.774 & 0.773 & 0.7608 \\ \midrule
\textbf{Average} & & & & & \textbf{0.638} & \textbf{0.590} \\ \bottomrule
\end{tabular}
}
\caption{Test set performance metrics for each language using the best model according to the development dataset results.}
\label{tab:test_metrics}
\end{table*}


Our analysis revealed several important patterns:

\begin{itemize}
\item \textbf{High-resource vs. Low-resource Languages}: High-resource languages like English (0.821/0.818), Russian (0.880/0.880), and Hindi (0.842/0.849) consistently achieved the highest performance across all approaches. Low-resource languages like Tigrinya (0.397/0.342), Swahili (0.440/0.409), and Makhuwa (0.257/0.205) showed significantly lower performance, with F1 scores often below 0.5.

\item \textbf{Language Family Analysis}: We observed performance clusters within language families. Indo-European languages (English, German, Spanish, Romanian) consistently performed well, with an average F1-micro of 0.795. Niger-Congo languages (Yoruba, Igbo, Swahili) showed more variable performance, with an average F1-micro of 0.563. Afro-Asiatic languages (Amharic, Tigrinya) generally performed below average, with F1-micro scores around 0.526.

\item \textbf{Retrieval Strategy Impact}: For some languages, particularly those with unique scripts or limited representation in pre-training data, n-gram retrieval outperformed embedding-based retrieval. This was evident in Oromo (0.607 vs. 0.552), Moroccan Arabic (0.576 vs. 0.552), and Chinese (0.748 vs. 0.698).

\item \textbf{Aggregation Strategy Effectiveness}: The label-specific F1 weighting strategy proved most effective overall, outperforming other aggregation methods in 17 out of 28 languages. This suggests that emotion-specific expertise varies across models and languages, and weighting predictions accordingly yields better results.
\end{itemize}

The performance gap between high-resource and low-resource languages remains a significant challenge. While our EmoRAG system reduced this gap compared to supervised approaches, substantial disparities persist. For instance, the F1-micro score for English (0.821) is more than three times that of Makhuwa (0.257), highlighting the continued challenges in multilingual emotion detection.

\subsection{Performance by Emotion}

% \begin{figure}[h]
%     \centering
%     \includegraphics[width=0.8\textwidth]{emotion_performance.png}
%     \caption{F1 scores by emotion across different approaches}
%     \label{fig:emotion_performance}
% \end{figure}

Our analysis of emotion-specific performance revealed consistent patterns across approaches:

\begin{itemize}
\item \textbf{Fear} was the most accurately detected emotion across all approaches, with the highest F1 scores (0.8182 for SetFit, 0.807 for BERT, 0.8036 for EmoRAG on English data). This may be due to the distinctive lexical markers associated with fear expressions.

\item \textbf{Anger} showed the most variable performance across approaches. BERT achieved an F1 of 0.5049, while EmoRAG reached 0.7879, suggesting that contextual understanding is particularly important for this emotion.

\item \textbf{Disgust} was the most challenging emotion to detect, with many languages lacking examples entirely. Where present, F1 scores were generally lower than for other emotions.

\item \textbf{Joy} and \textbf{Sadness} showed moderate performance across approaches, with EmoRAG consistently outperforming supervised methods (Joy: 0.8358 vs. 0.6227 for SetFit; Sadness: 0.8451 vs. 0.6491 for SetFit).

\item \textbf{Surprise} showed high variability across languages, with particularly low performance in German (0.2857) compared to English (0.7407).
\end{itemize}


\section{Ablation Studies}

To understand the contribution of different components to EmoRAG's performance, we conducted extensive ablation studies.

\subsection{EmoRAG Components}

\begin{table}[h]
\centering
\begin{tabular}{lcc}
\toprule
\textbf{Component Configuration} & \textbf{F1-micro} & \textbf{F1-macro} \\
\midrule
Full EmoRAG system & 0.678 & 0.634 \\
Without embedding retriever & 0.641 & 0.601 \\
Without n-gram retriever & 0.631 & 0.590 \\
Without label-specific weighting & 0.661 & 0.617 \\
Single model (GPT-4o-mini) & 0.631 & 0.590 \\
\bottomrule
\end{tabular}
\caption{Ablation study of EmoRAG components (averaged across all languages)}
\label{tab:ablation_components}
\end{table}

Removing the embedding-based retriever reduced performance by 5.5\% on average, with larger impacts on high-resource languages. Removing the n-gram retriever had a smaller overall impact (6.9\% reduction) but significantly affected low-resource languages with unique scripts. The label-specific weighting strategy contributed a 2.6\% improvement over simple majority voting, with larger gains in languages with imbalanced emotion distributions.

\subsection{Example Count Analysis}

% \begin{figure}[h]
%     \centering
%     \includegraphics[width=0.8\textwidth]{example_count_analysis.png}
%     \caption{Performance vs. number of retrieved examples}
%     \label{fig:example_count}
% \end{figure}

We varied the number of retrieved examples (K) from 0 to 1000 and found that performance generally improved with more examples up to a point, after which returns diminished or performance decreased. The optimal K varied by language: high-resource languages benefited from larger K values (100-150), while low-resource languages performed best with moderate K values (30-50). This suggests that quality of examples matters more than quantity for low-resource settings.

\begin{table}[h]
\centering
\begin{tabular}{lcc}
\toprule
\textbf{Number of Examples (K)} & \textbf{F1-micro} & \textbf{F1-macro} \\
\midrule
0 (zero-shot) & 0.5959 & 0.6164 \\
30 & 0.8036 & 0.8001 \\
100 & 0.8071 & 0.8026 \\
300 & 0.7911 & 0.7762 \\
1000 & 0.7882 & 0.7769 \\
\bottomrule
\end{tabular}
\caption{Impact of retrieved example count on GPT-4o-mini performance for English language}
\label{tab:example_count}
\end{table}

As shown in Table \ref{tab:example_count}, for English, performance improves dramatically when moving from zero-shot to few-shot with 30 examples, with a slight additional improvement at 100 examples. However, increasing to 300 or 1000 examples actually leads to decreased performance. This pattern suggests that there is an optimal window for the number of retrieved examples, beyond which the model may become overwhelmed with potentially conflicting information or less relevant examples.

\subsection{Cross-model Analysis}

\begin{table}[h]
\centering
\begin{tabular}{lcc}
\toprule
\textbf{Model} & \textbf{F1-micro} & \textbf{F1-macro} \\
\midrule
GPT-4o-mini & 0.631 & 0.590 \\
Llama-3.1-70B & 0.649 & 0.612 \\
Qwen2.5-72B & 0.657 & 0.621 \\
Gemma-2-27B & 0.618 & 0.583 \\
All models (ensemble) & 0.678 & 0.634 \\
\bottomrule
\end{tabular}
\caption{Performance comparison across different LLMs (averaged across all languages)}
\label{tab:cross_model}
\end{table}

Each model in our ensemble showed different strengths across languages and emotions. Llama-3.1-70B performed best on Germanic and Romance languages, while Qwen2.5-72B excelled on Asian languages. GPT-4o-mini showed more balanced performance across the board. Gemma-2-27B, despite its smaller size, contributed valuable predictions for specific emotion-language combinations, particularly for surprise and fear in Slavic languages.

The ensemble consistently outperformed individual models, with an average improvement of 3.2\% in F1-micro score over the best single model. This confirms the complementary nature of different LLMs' knowledge and the effectiveness of our aggregation strategies in leveraging their combined strengths.

\section{Discussion}

\subsection{Comparison of Approaches}
\begin{itemize}
\item \textbf{Strengths and Weaknesses}: Analysis of each approach
\item \textbf{Resource Efficiency}: Trade-offs between performance and computational requirements
\item \textbf{Practical Considerations}: Deployment scenarios for different approaches
\end{itemize}

\subsection{Performance on Low-resource Languages}
\begin{itemize}
\item \textbf{Challenges}: Specific issues in low-resource settings
\item \textbf{Transfer Learning Effects}: How well knowledge transfers to low-resource languages
\item \textbf{Retrieval Benefits}: How retrieval mechanisms help in low-resource scenarios
\end{itemize}

\subsection{Error Analysis}
\begin{itemize}
\item \textbf{Common Error Patterns}: Analysis of misclassifications
\item \textbf{Language-specific Challenges}: Linguistic factors affecting performance
\item \textbf{Cultural Factors}: Impact of cultural differences on emotion expression and detection
\end{itemize}

\section{Conclusion and Future Work}
\begin{itemize}
\item \textbf{Summary of Findings}: Key takeaways from the research
\item \textbf{Limitations}: Constraints and shortcomings of the current approaches
\item \textbf{Future Research Directions}: Promising avenues for further investigation
\item \textbf{Broader Impacts}: Potential applications and implications of the work
\end{itemize}

\printbibliography

\section*{Appendices}
\begin{itemize}
\item \textbf{A. Prompt Templates}: Complete prompts used for LLMs
\item \textbf{B. Hyperparameter Details}: Comprehensive listing of all parameters
\item \textbf{C. Additional Results}: Supplementary tables and figures
\item \textbf{D. Implementation Code}: Links to code repositories
\end{itemize}

\end{document}
